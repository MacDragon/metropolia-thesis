\documentclass[12pt,a4paper,oneside,article]{memoir}%Do not touch this first line ;)
% add hidelinks to documentclass to remove link boxes
% Global information (title of your thesis, your name, degree programme, major, etc.)

\def\bilingual{no}%For Finnish students, you must have 2 abstracts, one in English and one in your native language (Finnish or Swedish), so "yes", your thesis is bilingual.
%\def\bilingual{no}%For international student writing in English, only one language and one abstract.

%\def\thesislang{finnish} %change this depending on the main language of the thesis.
\def\thesislang{english} % "english" is the only other supported language currently. If someone has the swedish, please contribute!

\def\secondlang{english} %if the main language is Finnish (or Swedish), you must have 2 abstracts (one in Finnish (or Swedish) and one in English)
%\def\secondlang{finnish}
%If the main language is English and that you are native Finnish (or Swedish) speaker, you must have also abstract in your native language on top of the English one.

\author{Visa Harvey} %your first name and last name

%\def\alaotsikko{Alaotsikko/Subtitle} %DISABLED, seems not to be an option with the 2018 template. If you really need it, uncomment and modify style/title.tex accordingly.

%License
%When publishing your thesis to theseus.fi, you can keep all rights reserved to you,
%or use one of the Creative Commons https://creativecommons.org/licenses/?lang=en
%This attribute will set the license in the metadata of the generated pdf.
%possible options (case sensitive):
%all (keep all rights reserved to yourself)
%by (Attribution)
%by-sa (Attribution-ShareAlike)
%by-nd (Attribution-NoDerivs)
%by-nc (Attribution-NonCommercial)
%by-nc-sa (Attribution-NonCommercial-ShareAlike)
%by-nc-nd (Attribution-NonCommercial-NoDerivs)
%Note that theseus do not accept dedication to public domain CC0
\def\thesiscopy{all}

%Finnish section, for title/abstract
\def\otsikko{Opinnäytetyön otsikko}
\def\tutkinto{Tutkinto (esim. Insinööri (AMK))} % change to your needs, e.g. "YAMK", etc.
\def\kohjelma{Koulutusohjelma (esim. Tieto\textendash ja viestintätekniikka)}
\def\suuntautumis{Ammatillinen pääaine (esim. Mobile Solutions)}
\def\thesisfi{Insinöörityö}%was Opinnäytetyö
\def\ohjaajat{
Titteli Etunimi Sukunimi\newline
Titteli Etunimi Sukunimi
}
\def\tiivistelma{
Tämä on tiivistelmän ensimmäinen kappale. Tiivistelmän kappaleet loppuvat komentoon newline, jotta saadaan yksi tyhjä rivi aikaiseksi. \newline

Tämä on tiivistlemän toinen kappale.
}
\def\avainsanat{avainsana, avainsana}
\def\aihe{Lyhyt kuvaus opinnäytetyöstä. Max 255 merkkiä.}%for the pdf metadata/properties. If not used, empty it and also the \def\subject.

%English section, for title/abstract
\title{Automated device for Raspberry Pi Pico recovery}
\def\metropoliadegree{Bachelor of Engineering} % change to your needs, e.g. "master", etc.
\def\metropoliadegreeprogramme{Information Technology}
\def\metropoliaspecialisation{Smart Systems}
\def\thesisen{Bachelor’s Thesis} % change to your need, e.g. master's
\def\metropoliainstructors{
Keijo Länsikunnas, Principal Lecturer\newline
Anne Pajala, Assistant advisor
}
\def\abstract{
%TODO General note which I already mentioned. either expand or combine single sentence paragraphs.
The aim of this project is to create a standalone Raspberry Pi Pico recovery platform, able to initialise and recover a target Pi Pico development board to initial usable state or recover a MicroPython installation from a non accessible or booting startup script. The platform is to be created with a simple to use interface and a standalone method to update recovery images via any computer without utilising additional software.
\newline

The primary software functionality side of this thesis was achieved to a required level, with potential for further development to increase robustness. On the hardware side further development for target board interfacing and packaging into a standalone unit still needs to be researched and designed to achieve a fully complete practical solution.
\newline

}
\def\metropoliakeywords{SWD, Raspberry Pi Pico, USB Virtual File System}
\def\subject{short description of the thesis. Max 255 characters.}%for the pdf metadata/properties. If not used, empty it and also the \def\aihe.
