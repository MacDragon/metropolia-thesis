% Introduction

\chapter{Introduction}

Starting this year Metropolia University of Applied Sciences Information Technology course is moving to include a new first year module introducing embedded device programming involving learning with micropython before later modules delve to lower level C and C++.

Metropolia will be using commercially widely available and commodity cheap Raspberry Pi Pico or pico W \gls{mcu}  development boards to do this owing to their ease of use and wide range of available resources.

There will as a result be a lot of boards that need setting up ready for classes, clearing previous students saved files and otherwise ensuring they're ready for use by a new class. This currently would take quite a while having to manually boot each board into update mode when plugged in by holding down a boot select button to enter USB loader, and then potentially having to run multiple flash files to clear the board and install necessary firmware. This all takes valuable lecturer time. Further, faculty have identified an issue where auto booting Python code on the pi in micropythan can effectively 'brick' the board if boot code prevents the USB connection being opened from PC for communication. Normally to recover from this the whole board has to be cleared and re-flashed again.

This thesis aims to develop a self contained solution to automate management of these situations using a custom pi pico firmware to selectively re-flash a target board in a more automated fashion without needing to manually select boot modes or otherwise interact with the target board.
\pagebreak 