% Introduction

\chapter{Introduction}

Starting academic year 2022/23 Metropolia University of Applied Sciences Information Technology course is moving to include a new first year module introducing embedded device programming involving learning with MicroPython before later modules delve to lower level C and C++.

Metropolia will be using readily available widely available and commodity cheap Raspberry Pi Pico or Pico W \gls{mcu}  development boards to do this owing to their ease of use and wide range of available resources after searching for a replacement for discontinued LPCXpresso  LPC1549 \gls{mcu} development boards that have till now been used for many of the modules.

There will as a result be a lot of boards that need setting up ready for classes, clearing previous students saved files and otherwise ensuring they're ready for use by a new class. This currently would take quite a while having to manually boot each board into update mode when plugged in by holding down a boot select button to enter \gls{usb} loader, and then potentially having to run multiple flash files to clear the board and install necessary firmware. This all takes valuable lecturer time, faculty have also identified a notable issue where auto booting Python code on the pi in MicroPython can effectively 'brick' the board if boot code prevents the \gls{usb} connection being opened from PC for communication. Normally to recover from this the whole board has to be cleared and re-flashed again.

This thesis aims to develop a self contained solution to automate management of these situations using a custom pi pico firmware to selectively re-flash a target board in a more automated fashion without needing to manually select boot modes or otherwise interact with the target board.
\pagebreak 