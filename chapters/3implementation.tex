% Project Specifications
\clearpage%if the chapter heading starts close to bottom of the page, force a line break and remove the vertical vspace
\vspace{21.5pt}
\chapter{Implementation}
\section{Software  setup}

First state of project was getting initial programming environment ready, consisting of rp2040 SDK, Arm GCC cross compiler and Visual Studio Code\autoref{fig:vscode} with plugins for debugging and managing the rp2040 CMAKE build process, and a j-link SWD debugger to program and debug on the \gls{mcu} more flexibly than \gls{usb} UF2 flash programming. A logic analyser was also used in initial phases to help monitor the debug traffic for verification of handshake operations till communication was established between boards.

\begin{figure}[ht]
	\centering
	\AltText{Project loaded into Visual Studio Code}{\includegraphics[width=\textwidth]{vscode}}
	\caption{Visual Studio Code}
	\label{fig:vscode}
\end{figure}

\clearpage
\section{Hardware setup}
%

For first testing of compilation environment as initial proof of concept test bench setup a Pi Pico was directly wired into a J-Link debugger for simplicity as show in \autoref{fig:InitialPOCwiring}.

\begin{figure}[ht]
	\centering
	\AltText{Initial proof of concept hardware setup showing direct soldered wiring between J-Link Edu and a Raspberry pi Pico}{\includegraphics[width=\textwidth]{initial POC setup}}
	\caption{initial POC setup}
	\label{fig:InitialPOCwiring}
\end{figure}

Segger RTT\cite{JLinkRTTReal} is used for debug console access during testing to eliminate need for a seperate UART/Serial receiver and associated wiring.

In the final test setup a jumper cable is used to simulate the closing of a mounting interface interlock to similate plugging and unplugging target board without physically removing jumper wires from bench setup.

\pagebreak
This was later expanded to include a Pimoroni display and buttons for a user interface, and second Pi Pico as flashing target \autoref{fig:FullPOCwiring}, wiring diagram \autoref{fig:FullPOCwiringDiagram}

\begin{figure}[ht]
	\centering
	\AltText{Full POC hardware setup showing display and buttons for UI on breadboads along with target device}{\includegraphics[width=\textwidth]{full POC setup}}
	\caption{Full wiring on breadboard with Interface}
	\label{fig:FullPOCwiring}
\end{figure}

\begin{figure}[ht]
	\centering
	\AltText{Ki-Cad wiring schematic of POC setup}{\includegraphics[width=\textwidth]{POC wiring schematic}}
	\caption{Wiring diagram for bench test hardware setup}
	\label{fig:FullPOCwiringDiagram}
\end{figure}
\clearpage
\section{Software bootstrap}
nInitial step was to establish \gls{swd} setup as lowest level critical component, this took longer than anticipated after initial response reading DPIDR register succeeded to verify an \gls{arm} debug unit was responding, but further attempted operations such as memory read failed. Reason for failure transpired to a lack of accounting multiple debug units\cite{raspberrypiltdRaspberryPiPico} with many references to \gls{swd} implementation omitting the step of device selection from the setup sequence owing to most \gls{arm} \gls{mcu}'s only having one core this being an uncessary step.

Once handshake was established the most critical functions of memory write and read were implemented and verified to confirm successful control over \gls{swd}, 

From there a small \gls{ram} resident test program to blink the onboard \gls{led} was created as a test case binary to be write and executed from memory via \gls{swd} for a simple control verification, initially merely blinking \gls{led} to confirm code execution before building a shared memory structure for executing commands such as file recovery or flash programming/erasing.

This uses a \gls{magic} and runtime calculated CRC32 checksum for firmware data validation of transferred data during flashing process, and writing result codes back to memory that can be read by recovery device.

Programming sequence example to bootstrap running code on the target can be found in Appendix 1.

It was determined that 'bit banging' the \gls{swd} bus was sufficiently performant to not need a more complex implementation, the necessary 40kB helper image only takes 51ms to send over the bus which is inconsequential, and for  image upload operations the \gls{spi} flashing process is slower than the data upload  \gls{swd} speed, and most of the difference can be alleviated by using two data transfer locations to allow data upload whilst flashing is progressing in an interleaved write/transfer operation alternating locations.

This was measured to have approximately a 33\% speedup \autoref{table:flashspeed} in flashing versus using a single transfer location and waiting for write to complete before transferring new data.

\begin{table}[h]
	\centering
	\caption{Flashing speed with single and dual transfers}%IMPORTANT the caption must be before the tabular, so it will be on top of the table (there are other tricks to force it on top; but this one is straightforward).
	\vspace{-16.5pt}%time to time, spacing between caption and table can go too big...
	
	\begin{tabular}{|l|l|}
		\hline
		Flashing methodology & time for microPython 1.19 Image  \\ \hline
		One queued & 5562ms \\ \hline
		Dual queued interleave & 4154ms \\ \hline
	\end{tabular}
	\label{table:flashspeed}
\end{table}

To take control of target board, debug unit is first initialised to the rescue unit where a reset is performed, before switching over to regular debug unit and halting processor to ensure that the system is at a known state before uploading the  included sub helper binary image to handle flashing and recovery operations.

\clearpage
\section{File system implementation}

Due to the large amounts of repeated information and empty space within a UF2 format file direct storage into the pico's flash was impractical, especially with the effective requirement to store separate images for standard pico and W variations, which added together can reach a size bigger than the whole flash storage.

To enable maximum practical storage a more compact storage solution was implemented, consisting of three data areas for the UF2 image.

Laid out as follows: Data area, consisting of all the 256byte data packets from each UF2 block, An address area, indexing each of these data packets to their destination address and finally a header block where the filename, and other related metadata such as block count are stored. UF2 block /gls{crc} values were not stored and recalculated at runtime. Though a single checksum of all the data should still be added for verification before start of flashing.

Due to the requirement to store two images, there are two of these data structures kept, with the W one allocated the bulk of the storage to allow for the wireless module firmware and drivers taking up notably more storage than the regular pico as presented in table \ref{table:file_storage}:

\begin{table}[h]
	\centering
	\caption{Flash Storage layout}%IMPORTANT the caption must be before the tabular, so it will be on top of the table (there are other tricks to force it on top; but this one is straightforward).
	\vspace{-16.5pt}%time to time, spacing between caption and table can go too big...

		\begin{tabular}{|l|l|l|}
			\hline
			Address & Size  & Contents        \\ \hline
			0x000000 & 256kB & Program + Helper binary \\ \hline
			0x040000 & 4kB & UF2 Header 1      \\ \hline
			0x043000 & 12kB & UF2 Addresses 1 \\ \hline
			0x044000 & 600kB & UF2 Data 1 \\ \hline
			0x0D9000 & 4kB & UF2 Header 1 \\ \hline
			0x0DA000 & 20kB & UF2 Addresses 2      \\ \hline
			0x0DE000 & 1152kB & UF2 Data 2 \\ \hline
			0x1FFFFF & & End of flash. \\ \hline
		\end{tabular}
		\label{table:file_storage}
\end{table}

The \gls{usb} Virtual File System presented for UF2 image storage is based upon the TinyUF2 bootloader\cite{TinyUF2Bootloader2023}, which has been repurposed to capture file system uf2 transfer and store it to the internal UF2 storage.

To keep the files consistent even if power loss were to happen the file data is cleared at start of transaction, and header only written back once the full data is stored. Thus a power loss or other unexpected event mid transfer will leave the board with an empty file slot instead of inconsistent data.

One of the notable changes made to the TinyUF2 bootloader is to capture the file system information more fully, in order to read and store the full long file name of the UF2 image so that this can be presented to the user as an aid memoir on what firmware is loaded. This filename is also dynamically written to the virtual file system upon USB connection to present the data there also.

This required cacheing more of the \gls{fat} table than previously needed, due to long file names potentially requiring more than first cluster, especially with operating systems such as macOS that can also write other hidden helper files to the file system that need to be ignored.

Another modification over TinyUF2 loader is for the virtual image to present the currently stored image on the presented file system in recreated form when \gls{usb} loading is active including the current stored image size instead of presenting full flash image as originally designed, allowing end user to retrieve the image back off device as well as replace the stored image with an updated one.

\pagebreak
\section{Testing and UI}

For testing functionality, various likely scenarios were setup and verified to be repeatable.

First tests consisted of setting up recovery scenarios on MicroPython via Thonny\cite{ThonnyPythonIDE}, creating startup python scripts that both block the USB port or just run, and performing a recovery procedure to verify that the file is renamed successfully, or deleted, if a previous recovery result is still present.

Second tests verified that a populated file system could be wiped on an existing MicroPython install.

Third tests verified loading of different UF2 firmware files to the recovery platform, and then re imaging a target pico with the loaded firmware.

\clearpage
\subsection{USB Loading}
Figures \autoref{fig:picousbnodata} to \autoref{fig:usb_nodata_picow} show computer side of image loading in use, noting how the file system indicates which target device's firmware is being updated via use of dynamic text file name.

\begin{figure}[ht]
	\centering
	\AltText{File system shown on file system with no image loaded to device}{\includegraphics[width=\textwidth]{usb_nodata_pico}}
	\caption{Pico USB image loading, no data}
	\label{fig:picousbnodata}
\end{figure}

\begin{figure}[ht]
	\centering
	\AltText{File system shown on file system with an image loaded to device}{\includegraphics[width=\textwidth]{usb_data_pico}}
	\caption{Pico USB image loading, microPython image loaded}
	\label{fig:picousbdata}
\end{figure}

\begin{figure}[ht]
	\centering
	\AltText{File system shown on file system with no image loaded to device}{\includegraphics[width=\textwidth]{usb_nodata_picow}}
	\caption{Pico USB image loading, no data}
	\label{fig:usb_nodata_picow}
\end{figure}

\clearpage
\subsection{Recovery Process}
Whilst \autoref{fig:disp_startup_empty} to \autoref{fig:disp_fs_recovered} show user interface on device during states of initial empty recovery device, during \gls{usb} firmware loading, and both file system recovery and re imaging target device.

\begin{figure}[ht]
	\centering
	\AltText{Display startup no images stored}{\includegraphics[width=.7\textwidth]{disp_startup_empty}}
	\caption{Display startup no images stored}
	\label{fig:disp_startup_empty}
\end{figure}

\begin{figure}[ht]
	\centering
	\AltText{Display startup Pico image stored}{\includegraphics[width=.7\textwidth]{disp_startup_pico}}
	\caption{Display startup Pico image stored}
	\label{fig:disp_startup_pico}
\end{figure}

\begin{figure}[ht]
	\centering
	\AltText{Display showing recovery options menu}{\includegraphics[width=.7\textwidth]{disp_menu}}
	\caption{Display showing recovery options menu}
	\label{fig:disp_menu}
\end{figure}

\begin{figure}[ht]
	\centering
	\AltText{Display showing usb image upload in progress at 58\%}{\includegraphics[width=.7\textwidth]{disp_usbload}}
	\caption{Display showing usb image upload in progress}
	\label{fig:disp_usbload}
\end{figure}

\begin{figure}[ht]
	\centering
	\AltText{Display showing usb image having been uploaded}{\includegraphics[width=.7\textwidth]{disp_usbloaded}}
	\caption{Display showing usb image after successful upload}
	\label{fig:disp_usbloaded}
\end{figure}

\begin{figure}[ht]
	\centering
	\AltText{Display showing image flashing in progress}{\includegraphics[width=.7\textwidth]{disp_flashwrite}}
	\caption{Display showing image flashing to target in progress}
	\label{fig:disp_flashwrite}
\end{figure}

\begin{figure}[ht]
	\centering
	\AltText{Display showing filesystem recovered}{\includegraphics[width=.7\textwidth]{disp_fs_recovered}}
	\caption{Display showing successfully recovered filesystem}
	\label{fig:disp_fs_recovered}
\end{figure}


